% Marko Ristic

\documentclass[conference]{IEEEtran}

% 
% 8888888 888b     d888 8888888b.   .d88888b.  8888888b. 88888888888 .d8888b.  
%   888   8888b   d8888 888   Y88b d88P" "Y88b 888   Y88b    888    d88P  Y88b 
%   888   88888b.d88888 888    888 888     888 888    888    888    Y88b.      
%   888   888Y88888P888 888   d88P 888     888 888   d88P    888     "Y888b.   
%   888   888 Y888P 888 8888888P"  888     888 8888888P"     888        "Y88b. 
%   888   888  Y8P  888 888        888     888 888 T88b      888          "888 
%   888   888   "   888 888        Y88b. .d88P 888  T88b     888    Y88b  d88P 
% 8888888 888       888 888         "Y88888P"  888   T88b    888     "Y8888P"  
%                                                                              
%                                                                              
%                                                                              
% 

% Graphics
\usepackage{tikz}
\usepackage{graphicx}
% Maths
\usepackage{amsmath}
\usepackage{amsfonts}
\usepackage{amsthm}
\usepackage{mathtools}
% Algorithms
\usepackage{algorithm}
\usepackage{algpseudocode}
% Local
\usepackage{ISASmacros/isasmathmacros}
% Equal last columns
%\usepackage{flushend}


% Environments
\theoremstyle{definition}
\newtheorem{definition}{Definition}[section]

\theoremstyle{definition}
\newtheorem{theorem}{Theorem}[section]

\theoremstyle{remark}
\newtheorem*{remark}{Remark}

% Algorithm settings
\algnewcommand{\LineComment}[1]{\State \(\triangleright\) #1}

% Tikz settings
\usetikzlibrary{math}

% Tikz reused pics
\tikzset{
   plane/.pic = {
    \draw[fill]  plot[smooth, tension=0.6] coordinates {
        (-0.65,-0.9) 
        (-0.6,-0.85) 
        (-0.4,-0.75) 
        (-0.25,-0.65) 
        (-0.15,-0.5) 
        (-0.12,-0.3) 
        (-0.1,-0.1) 
        (0,0) 
        (0.1,-0.1) 
        (0.12,-0.3) 
        (0.15,-0.5) 
        (0.25,-0.65) 
        (0.4,-0.75) 
        (0.6,-0.85) 
        (0.65,-0.9)
        } -- plot[smooth, tension=0.6] coordinates {
        (0.65,-0.9) 
        (0.15,-0.91)
        (0.35,-1.1) 
        (0.37,-1.15)
        } -- plot[smooth, tension=0.6] coordinates {
        (0.37,-1.15)
        (0,-1.12) 
        (-0.37,-1.15) 
        } -- plot[smooth, tension=0.6] coordinates {
        (-0.37,-1.15)
        (-0.35,-1.1) 
        (-0.15,-0.91) 
        (-0.65,-0.9) 
        } -- cycle;
   }
}

% correct bad hyphenation here
\hyphenation{op-tical net-works semi-conduc-tor}


\begin{document}
% paper title
\title{Cryptographically Privileged State Estimation With Gaussian Keystreams}


% author names and affiliations
\author{\IEEEauthorblockN{Michael Shell}
\IEEEauthorblockA{School of Electrical and\\Computer Engineering\\
Georgia Institute of Technology\\
Atlanta, Georgia 30332--0250\\
Email: http://www.michaelshell.org/contact.html}
\and
\IEEEauthorblockN{Homer Simpson}
\IEEEauthorblockA{Twentieth Century Fox\\
Springfield, USA\\
Email: homer@thesimpsons.com}
\and
\IEEEauthorblockN{James Kirk\\ and Montgomery Scott}
\IEEEauthorblockA{Starfleet Academy\\
San Francisco, California 96678--2391\\
Telephone: (800) 555--1212\\
Fax: (888) 555--1212}}

% make the title area
\maketitle

% 
%        d8888 888888b.    .d8888b. 88888888888 8888888b.         d8888  .d8888b. 88888888888 
%       d88888 888  "88b  d88P  Y88b    888     888   Y88b       d88888 d88P  Y88b    888     
%      d88P888 888  .88P  Y88b.         888     888    888      d88P888 888    888    888     
%     d88P 888 8888888K.   "Y888b.      888     888   d88P     d88P 888 888           888     
%    d88P  888 888  "Y88b     "Y88b.    888     8888888P"     d88P  888 888           888     
%   d88P   888 888    888       "888    888     888 T88b     d88P   888 888    888    888     
%  d8888888888 888   d88P Y88b  d88P    888     888  T88b   d8888888888 Y88b  d88P    888     
% d88P     888 8888888P"   "Y8888P"     888     888   T88b d88P     888  "Y8888P"     888     
%                                                                                             
%                                                                                             
%                                                                                             
% 

\begin{abstract}
The abstract goes here.
\end{abstract}

% no keywords

% For peerreview papers, this IEEEtran command inserts a page break and
% creates the second title. It will be ignored for other modes.
\IEEEpeerreviewmaketitle

% 
% 8888888 888b    888 88888888888 8888888b.   .d88888b.  
%   888   8888b   888     888     888   Y88b d88P" "Y88b 
%   888   88888b  888     888     888    888 888     888 
%   888   888Y88b 888     888     888   d88P 888     888 
%   888   888 Y88b888     888     8888888P"  888     888 
%   888   888  Y88888     888     888 T88b   888     888 
%   888   888   Y8888     888     888  T88b  Y88b. .d88P 
% 8888888 888    Y888     888     888   T88b  "Y88888P"  
%                                                        
%                                                        
%                                                        
% 

\section{Introduction}
% no \IEEEPARstart
Temporary token reference \cite{katzIntroductionModernCryptography2008}. Start:

State estimation

Wireless and distributed estimation

Security concerns

Traditional methods hide all information, other use-cases exist

Information may be divided into privilege levels authenticating different audiences to different amounts of information [gps, anonymisation]

Contribution...

Section summary

\subsection{Notation}
Define vectors, matrices, encryption, pseudorandom samples, positive-definitness and $\prec$ for matrices


% 
% 8888888b.  8888888b.   .d88888b.  888888b.   
% 888   Y88b 888   Y88b d88P" "Y88b 888  "88b  
% 888    888 888    888 888     888 888  .88P  
% 888   d88P 888   d88P 888     888 8888888K.  
% 8888888P"  8888888P"  888     888 888  "Y88b 
% 888        888 T88b   888     888 888    888 
% 888        888  T88b  Y88b. .d88P 888   d88P 
% 888        888   T88b  "Y88888P"  8888888P"  
%                                              
%                                              
%                                              
% 

\section{Problem Statement}
Linear time-invariant system

Kalman filter equations 

KF meets the theoretical best estimator in terms of mean square error as evident from the CRLB [estimation book]

The aim is to produce measurements such that estimators have different estimation lower bounds depending on their knowledge of a shared secret key with the sensor. This is in accordance with the Kirchoff principle [crypto book] and reduces all secrecy to a single, replaceable, uniformly random integer.

We will refer to the estimator holding a shared key with the sensor as a privileged estimator, and the one without, an unprivileged estimator.


% 
% 8888888b.  8888888b.  8888888 888     888      8888888888 .d8888b. 88888888888 
% 888   Y88b 888   Y88b   888   888     888      888       d88P  Y88b    888     
% 888    888 888    888   888   888     888      888       Y88b.         888     
% 888   d88P 888   d88P   888   Y88b   d88P      8888888    "Y888b.      888     
% 8888888P"  8888888P"    888    Y88b d88P       888           "Y88b.    888     
% 888        888 T88b     888     Y88o88P        888             "888    888     
% 888        888  T88b    888      Y888P         888       Y88b  d88P    888     
% 888        888   T88b 8888888     Y8P          8888888888 "Y8888P"     888     
%                                                                                
%                                                                                
%                                                                                
% 

\section{Privileged Estimation}
General idea

(picture ?)

Use a cryptographically secure key stream to generate pseudorandom Gaussian samples

Samples are used to increase the uncertainty of estimation and are known and removable only by those with the key used to generate them

\subsection{Gaussian Keystream}
To generate pseudorandom Gaussian samples, we rely on first generating a traditional pseudorandom bitstream given a secret key.

Using well-studied methods for the generation of pseudorandomness guarantees robustness and an easy means of updating only the relevant component when the methods used are no longer considered safe.

Any implementation of a cryptographic stream cipher can be used for our purpose and will produce a stream of bits typically combined with plaintexts to provide secure encryption.

Rather than encrypting plaintext, we interpret the bitstream as sequential pseudorandom integers and use these to generate pseudorandom uniform real numbers in the range (0,1). $u$

While the uniform real samples are only approximated by floating-point numbers in the conversion from integers we argue this is sufficiently uniform and discuss this further in the Security section.

Finally, independent standard Gaussian samples can then be generated from the uniform real numbers using the Box-Muller transform, and are ready to be used by our sensor and privileged filter. $z$

\subsection{Additional Gaussian Noise}
To use the pseudorandom Gaussian samples at the sensor and privileged estimator, they need to be converted to multivariate Gaussian samples suitable for use in the measurement model and need a means of controlling how much uncertainty is added to the unprivileged estimators.

We define the additional noise term $Z>0$ and can transform the Gaussian samples $z$ into pseudorandom samples $o$ of a multivariate zero-mean Gaussian distribution with covariance $Z$.

Prior to estimation, we assume that a secret key is shared between the sensor and the privileged estimator.

During estimation, the sensor modifies its measurements at each timestep

In the case where pseudorandomness is indistinguishable from randomness, as is the case at an unprivileged estimator when using cryptographically sounds keystreams for pseudorandomness, the measurement model can now been written as $R+Z$

For the privileged estimator who holds the shared secret key, values $z$ and therefore $o$ can be computed at any time $k$ and recieved measurements modified to their original form. This inturn results in exactly the measurement model from the problem formulation.


crlb here somewhere?


\subsection{Multiple Privileges}


\section{Scheme Security}

\subsection{Single Additional Noise}

\subsection{Multiple Additional Noises}


\section{Simulation and Results}


% An example of a floating figure using the graphicx package.
% Note that \label must occur AFTER (or within) \caption.
% For figures, \caption should occur after the \includegraphics.
% Note that IEEEtran v1.7 and later has special internal code that
% is designed to preserve the operation of \label within \caption
% even when the captionsoff option is in effect. However, because
% of issues like this, it may be the safest practice to put all your
% \label just after \caption rather than within \caption{}.
%
% Reminder: the "draftcls" or "draftclsnofoot", not "draft", class
% option should be used if it is desired that the figures are to be
% displayed while in draft mode.
%
%\begin{figure}[!t]
%\centering
%\includegraphics[width=2.5in]{myfigure}
% where an .eps filename suffix will be assumed under latex, 
% and a .pdf suffix will be assumed for pdflatex; or what has been declared
% via \DeclareGraphicsExtensions.
%\caption{Simulation results for the network.}
%\label{fig_sim}
%\end{figure}

% Note that the IEEE typically puts floats only at the top, even when this
% results in a large percentage of a column being occupied by floats.


% An example of a double column floating figure using two subfigures.
% (The subfig.sty package must be loaded for this to work.)
% The subfigure \label commands are set within each subfloat command,
% and the \label for the overall figure must come after \caption.
% \hfil is used as a separator to get equal spacing.
% Watch out that the combined width of all the subfigures on a 
% line do not exceed the text width or a line break will occur.
%
%\begin{figure*}[!t]
%\centering
%\subfloat[Case I]{\includegraphics[width=2.5in]{box}%
%\label{fig_first_case}}
%\hfil
%\subfloat[Case II]{\includegraphics[width=2.5in]{box}%
%\label{fig_second_case}}
%\caption{Simulation results for the network.}
%\label{fig_sim}
%\end{figure*}
%
% Note that often IEEE papers with subfigures do not employ subfigure
% captions (using the optional argument to \subfloat[]), but instead will
% reference/describe all of them (a), (b), etc., within the main caption.
% Be aware that for subfig.sty to generate the (a), (b), etc., subfigure
% labels, the optional argument to \subfloat must be present. If a
% subcaption is not desired, just leave its contents blank,
% e.g., \subfloat[].


% An example of a floating table. Note that, for IEEE style tables, the
% \caption command should come BEFORE the table and, given that table
% captions serve much like titles, are usually capitalized except for words
% such as a, an, and, as, at, but, by, for, in, nor, of, on, or, the, to
% and up, which are usually not capitalized unless they are the first or
% last word of the caption. Table text will default to \footnotesize as
% the IEEE normally uses this smaller font for tables.
% The \label must come after \caption as always.
%
%\begin{table}[!t]
%% increase table row spacing, adjust to taste
%\renewcommand{\arraystretch}{1.3}
% if using array.sty, it might be a good idea to tweak the value of
% \extrarowheight as needed to properly center the text within the cells
%\caption{An Example of a Table}
%\label{table_example}
%\centering
%% Some packages, such as MDW tools, offer better commands for making tables
%% than the plain LaTeX2e tabular which is used here.
%\begin{tabular}{|c||c|}
%\hline
%One & Two\\
%\hline
%Three & Four\\
%\hline
%\end{tabular}
%\end{table}


% Note that the IEEE does not put floats in the very first column
% - or typically anywhere on the first page for that matter. Also,
% in-text middle ("here") positioning is typically not used, but it
% is allowed and encouraged for Computer Society conferences (but
% not Computer Society journals). Most IEEE journals/conferences use
% top floats exclusively. 
% Note that, LaTeX2e, unlike IEEE journals/conferences, places
% footnotes above bottom floats. This can be corrected via the
% \fnbelowfloat command of the stfloats package.



% 
%  .d8888b.   .d88888b.  888b    888  .d8888b.  
% d88P  Y88b d88P" "Y88b 8888b   888 d88P  Y88b 
% 888    888 888     888 88888b  888 888    888 
% 888        888     888 888Y88b 888 888        
% 888        888     888 888 Y88b888 888        
% 888    888 888     888 888  Y88888 888    888 
% Y88b  d88P Y88b. .d88P 888   Y8888 Y88b  d88P 
%  "Y8888P"   "Y88888P"  888    Y888  "Y8888P"  
%                                               
%                                               
%                                               
% 

\section{Conclusion}
The conclusion goes here.

% conference papers do not normally have an appendix

% use section* for acknowledgment
\section*{Acknowledgment}
The authors would like to thank...

% Bibliography
\bibliographystyle{IEEEtran}
\bibliography{BibTeX/PrivilegedEstimation}

% that's all folks
\end{document}


